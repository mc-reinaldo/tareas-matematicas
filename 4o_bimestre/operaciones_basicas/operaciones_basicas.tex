\newcommand{\seccion}{SECUNDARIA INCORPORADA A LA SEG }
\newcommand{\descripcion}{Operaciones b\'asicas}
\newcommand{\grado}{Primero de secundaria}
\newcommand{\ciclo}{Ciclo escolar: 2015--2016}
\newcommand{\papel}{letterpaper} %letterpaper, legalpaper ...
\newcommand{\fecha}{14 de marzo de 2016}

\author{M. en C. Reinaldo Zapata}

\documentclass[11pt]{article}
\usepackage[\papel]{geometry}

\title{\vspace{-1cm}\flushleft \seccion \\ \descripcion \\  \grado \\ \ciclo}

\newcommand\BackgroundLogo{
\put(160,285){
\parbox[b][\paperheight]{\paperwidth}{%
\vfill
\centering
\includegraphics[width=5cm,height=2.5cm,keepaspectratio]{/Users/reinaldo/Documents/clases/jassa/logo}%
\vfill
}}}

% \hyphenation{con-ti-nua-ci\'on}

\usepackage{enumitem}
\usepackage[T1]{fontenc} %fuentes
\usepackage{lmodern} %fuente mejorada
\usepackage[spanish]{babel}
\decimalpoint
\usepackage{fullpage}
\usepackage{graphicx}
\usepackage{eso-pic}
\usepackage{multirow}
\usepackage{subfigure}
\usepackage{tikz}
\usepackage{hyperref} 
\usepackage{color}
\usepackage{multicol}
\usepackage{tikz}
\usetikzlibrary{shapes.geometric}

\setlist[enumerate]{itemsep=-1.5mm}


\usepackage[leqno,fleqn]{amsmath}
\makeatletter
  \def\tagform@#1{\maketag@@@{#1\@@italiccorr}}
\makeatother
\renewcommand{\theequation}{\fbox{\textbf{\arabic{equation}}}}


\begin{document}
\AddToShipoutPicture*{\BackgroundLogo}
\ClearShipoutPicture
\date{\fecha}
\maketitle


% \begin{minipage}[t]{0.8\linewidth}
Nombre del alumno:\,\line(1,0){244}\,.\hspace*{.2cm} \hfill Aciertos:\,\line(1,0){35}\,. \\
\indent Primero de secundaria, grupo:\,\line(1,0){35}\,. No. de lista:\,\line(1,0){35}\,. \hfill 10 \quad \ 
% \end{minipage}
% \begin{minipage}{0.8\linewidth}
% \end{minipage}

\vspace{5mm}

Resuelve las siguientes operaciones b\'asicas respetando su jerarqu\'ia.
Recuerda incluir todas las operaciones para que tu tarea tenga validez. Usa tres
decimales en las divisiones.

\begin{multicols}{2}

\begin{equation}
\frac{3}{5}+6 \frac{1}{2} =
\end{equation}

\vspace{4cm}

\begin{equation}
\frac{7}{3}-1 \frac{1}{7} =
\end{equation}

\vspace{4cm}

\begin{equation}
(2.5)(3.1416) =
\end{equation}

\vspace{4cm}

\begin{equation}
2.4 \div 3.5 =
\end{equation}

\vspace{4cm}

\end{multicols}

\newpage

\begin{multicols}{2}

\begin{equation}
4.8 \div 1.12 =
\end{equation}

\vspace{4cm}

\begin{equation}
\left( \frac{3}{5}+ 2 \frac{1}{2} \right) \div 1\frac{2}{3} =
\end{equation}


\vspace{4cm}

\begin{equation}
\frac{3}{5}+ 2 \frac{1}{2} \div 1\frac{2}{3} =
\end{equation}

\vspace{4cm}

\begin{equation}
(1.5 + 4.6)(3 - 1.9)  =
\end{equation}

\vspace{4cm}

\begin{equation}
7.8 - 1.009 = 
\end{equation}

\vspace{4cm}

\begin{equation}
10 - 0.8976 =
\end{equation}

\vspace{4cm}

\end{multicols}

\vspace{4cm}


\noindent
Extra: usa fracciones para resolver 
\begin{equation}
1.8 + 4\frac{1}{3} - 1.2 =
\end{equation}

\end{document}

