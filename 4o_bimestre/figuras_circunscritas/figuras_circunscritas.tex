\newcommand{\seccion}{SECUNDARIA INCORPORADA A LA SEG }
\newcommand{\descripcion}{\sc Figuras regulares circunscritas}
\newcommand{\grado}{Primero de secundaria}
\newcommand{\ciclo}{Ciclo escolar: 2015--2016}
\newcommand{\papel}{letterpaper} %letterpaper, legalpaper ...
\newcommand{\fecha}{7 de abril de 2016}

\author{M. en C. Reinaldo Zapata}

\documentclass[11pt]{article}
\usepackage[\papel]{geometry}

\title{\vspace{-1cm}\flushleft \seccion \\ \descripcion \\  \grado \\ \ciclo}

\newcommand\BackgroundLogo{
\put(160,285){
\parbox[b][\paperheight]{\paperwidth}{%
\vfill
\centering
\includegraphics[width=5cm,height=2.5cm,keepaspectratio]{/Users/reinaldo/Documents/clases/jassa/logo}%
\vfill
}}}

% \hyphenation{con-ti-nua-ci\'on}

\usepackage{enumitem}
\usepackage[T1]{fontenc} %fuentes
\usepackage{lmodern} %fuente mejorada
\usepackage[spanish]{babel}
\decimalpoint
\usepackage{fullpage}
\usepackage{graphicx}
\usepackage{eso-pic}
\usepackage{multirow}
\usepackage{subfigure}
\usepackage{tikz}
\usepackage{hyperref} 
\usepackage{color}
\usepackage{multicol}
\usepackage{tikz}
\usetikzlibrary{shapes.geometric}



\usepackage[leqno,fleqn]{amsmath}
\makeatletter
  \def\tagform@#1{\maketag@@@{#1\@@italiccorr}}
\makeatother
\renewcommand{\theequation}{\fbox{\textbf{\arabic{equation}}}}


\begin{document}
\AddToShipoutPicture*{\BackgroundLogo}
\ClearShipoutPicture
\date{\fecha}
\maketitle

\vspace{-3mm}
% \begin{minipage}[t]{0.8\linewidth}
Nombre del alumno:\,\line(1,0){244}\,.\hspace*{.2cm} \hfill Aciertos:\,\line(1,0){35}\,. \\
\indent Primero de secundaria, grupo:\,\line(1,0){35}\,. No. de lista:\,\line(1,0){35}\,. \hfill 12 \quad \ 
% \end{minipage}
% \begin{minipage}{0.8\linewidth}
% \end{minipage}

\vspace{3mm}
Recuerda que para que tu trabajo tenga validez deber\'as incluir todas las
operaciones y procedimientos que realices.

\vspace{3mm}

Siguiendo el procedimiento visto en clase construye las figuras circunscritas
que se piden y marca el apotema en las \'ultimas dos. Calcula el per\'imetro y
el \'area de cada una de ellas; para ello usa tu regla y haz las medidas
necesarias.

\vspace{5mm}

Tri\'angulo circunscrito en una circunferencia de 6\,cm de radio.

\newpage

Pent\'agono circunscrito en una circunferencia de 4\,cm de radio.

\vfill

Hex\'agono circunscrito en una circunferencia de 10 cm de di\'ametro

\vfill

\

\end{document}










