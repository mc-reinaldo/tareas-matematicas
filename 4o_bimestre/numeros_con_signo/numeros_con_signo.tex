\newcommand{\seccion}{SECUNDARIA INCORPORADA A LA SEG }
\newcommand{\descripcion}{N\'umeros con Signo}
\newcommand{\grado}{Primero de secundaria}
\newcommand{\ciclo}{Ciclo escolar: 2015--2016}
\newcommand{\papel}{letterpaper} %letterpaper, legalpaper ...
\newcommand{\fecha}{3 de marzo de 2016}

\author{M. en C. Reinaldo Zapata}

\documentclass[11pt]{article}
\usepackage[\papel]{geometry}

\title{\vspace{-1cm}\flushleft \seccion \\ \descripcion \\  \grado \\ \ciclo}

\newcommand\BackgroundLogo{
\put(160,285){
\parbox[b][\paperheight]{\paperwidth}{%
\vfill
\centering
\includegraphics[width=5cm,height=2.5cm,keepaspectratio]{/Users/reinaldo/Documents/clases/jassa/logo}%
\vfill
}}}

% \hyphenation{con-ti-nua-ci\'on}

\usepackage{enumitem}
\usepackage[T1]{fontenc} %fuentes
\usepackage{lmodern} %fuente mejorada
\usepackage[spanish]{babel}
\decimalpoint
\usepackage{fullpage}
\usepackage{graphicx}
\usepackage{eso-pic}
\usepackage{multirow}
\usepackage{subfigure}
\usepackage{tikz}
\usepackage{hyperref} 
\usepackage{color}
\usepackage{multicol}
\usepackage{tikz}
\usetikzlibrary{shapes.geometric}

\setlist[enumerate]{itemsep=-1.5mm}


\usepackage[leqno,fleqn]{amsmath}
\makeatletter
  \def\tagform@#1{\maketag@@@{#1\@@italiccorr}}
\makeatother
\renewcommand{\theequation}{\fbox{\textbf{\arabic{equation}}}}


\begin{document}
\AddToShipoutPicture*{\BackgroundLogo}
\ClearShipoutPicture
\date{\fecha}
\maketitle


% \begin{minipage}[t]{0.8\linewidth}
Nombre del alumno:\,\line(1,0){244}\,.\hspace*{.2cm} \hfill Aciertos:\,\line(1,0){35}\,. \\
\indent Primero de secundaria, grupo:\,\line(1,0){35}\,. No. de lista:\,\line(1,0){35}\,. \hfill 10 \quad \ 
% \end{minipage}
% \begin{minipage}{0.8\linewidth}
% \end{minipage}

\vspace{5mm}

Recuerda qeu el procedimiento para resolver operaciones que implican n\'umeros
positivos y negativos es el siguiente:

\begin{enumerate}
\item Sumar todos los n\'umeros positivos.
\item Sumar todos los n\'umeros negativos.
\item Restar los resultados anteriores.
\item Seleccionar el signo adecuado para el resultado: si el valor de los
positivos es mayor que el de los negativos, el resultado ser\'a positivo; en
caso de no serlo, entonces ser\'a negativo.
\end{enumerate}

Siguiendo el procedimiento mencionado anteriormente, resuleve las siguientes
operaciones que implican n\'umeros positivos y negativos.

\begin{multicols}{2}

\begin{equation}
5+4-3+2-6 =
\end{equation}
\begin{equation}
-6+8-3+5 =
\end{equation}
\begin{equation}
5-10+8-3 =
\end{equation}
\begin{equation}
-7+6-5+4 =
\end{equation}
\begin{equation}
40-75+8-60=
\end{equation}
\begin{equation}
-200+100-400+1000 =
\end{equation}
\begin{equation}
6+8+7-6 =
\end{equation}
\begin{equation}
-5-2-5-29 =
\end{equation}
\begin{equation}
3.4-5.3+8.5-10 =
\end{equation}
\begin{equation}
-\frac{1}{3} + \frac{2}{4} \frac{1}{6} + \frac{7}{12} =
\end{equation}


\end{multicols}
    
\end{document}

