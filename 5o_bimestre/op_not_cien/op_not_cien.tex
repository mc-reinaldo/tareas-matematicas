\newcommand{\seccion}{SECUNDARIA INCORPORADA A LA SEG }
\newcommand{\descripcion}{\sc Operaciones con notaci\'on cient\'ifica}
\newcommand{\grado}{Primero de Secundaria}
\newcommand{\ciclo}{Ciclo escolar: 2015--2016}
\newcommand{\papel}{letterpaper} %letterpaper, legalpaper ...
\newcommand{\fecha}{7 de junio de 2016}

\author{M. en C. Reinaldo Zapata}

\documentclass[11pt]{article}
\usepackage[\papel]{geometry}

\title{\vspace{-1cm}\flushleft \seccion \\ \descripcion \\  \grado \\ \ciclo}

\newcommand\BackgroundLogo{
\put(160,285){
\parbox[b][\paperheight]{\paperwidth}{%
\vfill
\centering
\includegraphics[width=5cm,height=2.5cm,keepaspectratio]{/Users/reinaldo/Documents/clases/jassa/logo}%
\vfill
}}}

% \hyphenation{con-ti-nua-ci\'on}

\usepackage{enumitem}
\usepackage[T1]{fontenc} %fuentes
\usepackage{lmodern} %fuente mejorada
\usepackage[spanish]{babel}
\decimalpoint
\usepackage{fullpage}
\usepackage{graphicx}
\usepackage{eso-pic}
\usepackage{multirow}
\usepackage{subfigure}
\usepackage{tikz}
\usepackage{hyperref} 
\usepackage{color}
\usepackage{multicol}
\usepackage{tikz}
\usetikzlibrary{shapes.geometric}



\usepackage[leqno,fleqn]{amsmath}
\makeatletter
  \def\tagform@#1{\maketag@@@{#1\@@italiccorr}}
\makeatother
\renewcommand{\theequation}{\fbox{\textbf{\arabic{equation}}}}


\begin{document}
\AddToShipoutPicture*{\BackgroundLogo}
\ClearShipoutPicture
\date{\fecha}
\maketitle

\vspace{-3mm}
% \begin{minipage}[t]{0.8\linewidth}
Nombre del alumno:\,\line(1,0){244}\,.\hspace*{.2cm} \hfill Aciertos:\,\line(1,0){35}\,. \\
\indent Primero de secundaria, grupo:\,\line(1,0){35}\,. No. de lista:\,\line(1,0){35}\,. \hfill 10 \quad \ 
% \end{minipage}
% \begin{minipage}{0.8\linewidth}
% \end{minipage}

\vspace{3mm}
Recuerda que para que tu trabajo tenga validez deber\'as incluir todas las
operaciones y procedimientos que realices.

\section*{Operaciones} % (fold)
\label{sec:operaciones}

Resuelve las siguientes operaciones usnado notaci\'on cient\'ifica. Escribe el
resultado final utilizando s\'olo una posici\'on en los enteros.
\begin{equation}
21.24 \times 10^{-4} + 3.58 \times 10^{-5} + 5.69  \times 10^{-6} = 
\end{equation} 

\vspace{3cm}
\begin{equation}
149.35 \times 10^{-5} - 21.8 \times 10^{-6} - 13.84 \times 10^{-8} =
\end{equation}

\newpage
\vspace{3cm}
\begin{equation}
12.3 \times 10^{-1} - 6.35 \times 10^{-2} + 567 = 
\end{equation}

\vspace{3cm}
\begin{equation}
(7.85 \times 10^{15})(2.9 \times 10^{-3}) =
\end{equation}

\vspace{3cm}
\begin{equation}
2500 \times 10^{6} \div 2.5 \times 10^{-8} =
\end{equation}

\vspace{3cm}
\section*{puntos extra} % (fold)
\label{sec:puntos_extra}

\begin{equation}
(2 \times 10^{6})^{5} = 
\end{equation}
% section operaciones (end)
% section puntos_extra (end)

\vspace{3cm}


% section problemas (end)
\end{document}










