\newcommand{\seccion}{SECUNDARIA INCORPORADA A LA SEG }
\newcommand{\descripcion}{\sc Problemas con n\'umeros positivos \\y negativos}
\newcommand{\grado}{Primero de secundaria}
\newcommand{\ciclo}{Ciclo escolar: 2015--2016}
\newcommand{\papel}{letterpaper} %letterpaper, legalpaper ...
\newcommand{\fecha}{23 de mayo de 2016}

\author{M. en C. Reinaldo Zapata}

\documentclass[11pt]{article}
\usepackage[\papel]{geometry}

\title{\vspace{-1cm}\flushleft \seccion \\ \descripcion \\  \grado \\ \ciclo}

\newcommand\BackgroundLogo{
\put(160,285){
\parbox[b][\paperheight]{\paperwidth}{%
\vfill
\centering
\includegraphics[width=5cm,height=2.5cm,keepaspectratio]{/Users/reinaldo/Documents/clases/jassa/logo}%
\vfill
}}}

% \hyphenation{con-ti-nua-ci\'on}

\usepackage{enumitem}
\usepackage[T1]{fontenc} %fuentes
\usepackage{lmodern} %fuente mejorada
\usepackage[spanish]{babel}
\decimalpoint
\usepackage{fullpage}
\usepackage{graphicx}
\usepackage{eso-pic}
\usepackage{multirow}
\usepackage{subfigure}
\usepackage{tikz}
\usepackage{hyperref} 
\usepackage{color}
\usepackage{multicol}
\usepackage{tikz}
\usetikzlibrary{shapes.geometric}



\usepackage[leqno,fleqn]{amsmath}
\makeatletter
  \def\tagform@#1{\maketag@@@{#1\@@italiccorr}}
\makeatother
\renewcommand{\theequation}{\fbox{\textbf{\arabic{equation}}}}


\begin{document}
\AddToShipoutPicture*{\BackgroundLogo}
\ClearShipoutPicture
\date{\fecha}
\maketitle

\vspace{-3mm}
% \begin{minipage}[t]{0.8\linewidth}
Nombre del alumno:\,\line(1,0){244}\,.\hspace*{.2cm} \hfill Aciertos:\,\line(1,0){35}\,. \\
\indent Primero de secundaria, grupo:\,\line(1,0){35}\,. No. de lista:\,\line(1,0){35}\,. \hfill 10 \quad \ 
% \end{minipage}
% \begin{minipage}{0.8\linewidth}
% \end{minipage}

\vspace{3mm}
Resuelve los sigientes problemas referentes a operaciones de n\'umeros con signo.
Recuerda que para que tu trabajo tenga validez deber\'as incluir todas las
operaciones y procedimientos que realices.

\begin{enumerate}
\item El rey Nabucodonosor II fue el rey m\'as reconocido de la dinast\'ia
Caldea de Babilonia, lugar donde ahora se encuentra Irak. Naci\'o en el a\~no
630 a.C. y muri\'o en el a\~no 562 a.C. Su reinado fue desde el a\~no 605 al
562 a.C. Calcula a que edad muri\'o y el tiempo que dur\'o su reinado.

\vfill
\item El Imperio romano fue el tercer periodo de civilizaci\'on romana en la
Antig\"edad Cl\'asica. Dicho preriodo fue comprendido entre los a\~nos 27 a.C. y 
476 d.C. Determina cual fue la duraci\'on de dicho periodo. 

\vfill

\newpage
\item La civilizaci\'on maya fue una civilizaci\'on mesoamericana que destac\'o
en Am\'erica por su escritura gl\'ifica, el \'unico sistema de escritura
plenamente desarrollado del continente americano precolombino, as\'i como por su
arte, arquitectura y sistemas de matem\'atica y de astronom\'ia.

\vfill
\item La Biblia es el conjunto de libros can\'onicos del juda\'ismo y el
cristianismo. La canonicidad de cada libro var\'ia dependiendo de la tradici\'on
adoptada. Seg\'un las religiones jud\'ia y cristiana, transmite la palabra de
Dios. Hasta 2008, ha sido traducida a 2454 idiomas. Su per\'iodo de escritura
fue entre los a\~nos 900 a.C. y 100 d.C. ?`Cu\'anto dur\'o dicho per\'iodo de
escritura?

\vfill
\end{enumerate}

\end{document}










