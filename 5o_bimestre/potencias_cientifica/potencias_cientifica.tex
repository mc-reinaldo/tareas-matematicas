\newcommand{\seccion}{SECUNDARIA INCORPORADA A LA SEG }
\newcommand{\descripcion}{\sc Potencias y Notaci\'on Cient\'ifica}
\newcommand{\grado}{Primero de secundaria}
\newcommand{\ciclo}{Ciclo escolar: 2015--2016}
\newcommand{\papel}{letterpaper} %letterpaper, legalpaper ...
\newcommand{\fecha}{23 de mayo de 2016}

\author{M. en C. Reinaldo Zapata}

\documentclass[11pt]{article}
\usepackage[\papel]{geometry}

\title{\vspace{-1cm}\flushleft \seccion \\ \descripcion \\  \grado \\ \ciclo}

\newcommand\BackgroundLogo{
\put(160,285){
\parbox[b][\paperheight]{\paperwidth}{%
\vfill
\centering
\includegraphics[width=5cm,height=2.5cm,keepaspectratio]{/Users/reinaldo/Documents/clases/jassa/logo}%
\vfill
}}}

% \hyphenation{con-ti-nua-ci\'on}

\usepackage{enumitem}
\usepackage[T1]{fontenc} %fuentes
\usepackage{lmodern} %fuente mejorada
\usepackage[spanish]{babel}
\decimalpoint
\usepackage{fullpage}
\usepackage{graphicx}
\usepackage{eso-pic}
\usepackage{multirow}
\usepackage{subfigure}
\usepackage{tikz}
\usepackage{hyperref} 
\usepackage{color}
\usepackage{multicol}
\usepackage{tikz}
\usetikzlibrary{shapes.geometric}



\usepackage[leqno,fleqn]{amsmath}
\makeatletter
  \def\tagform@#1{\maketag@@@{#1\@@italiccorr}}
\makeatother
\renewcommand{\theequation}{\fbox{\textbf{\arabic{equation}}}}


\begin{document}
\AddToShipoutPicture*{\BackgroundLogo}
\ClearShipoutPicture
\date{\fecha}
\maketitle

\vspace{-3mm}
% \begin{minipage}[t]{0.8\linewidth}
Nombre del alumno:\,\line(1,0){244}\,.\hspace*{.2cm} \hfill Aciertos:\,\line(1,0){35}\,. \\
\indent Primero de secundaria, grupo:\,\line(1,0){35}\,. No. de lista:\,\line(1,0){35}\,. \hfill 10 \quad \ 
% \end{minipage}
% \begin{minipage}{0.8\linewidth}
% \end{minipage}

\vspace{3mm}
Resuelve las sigientes operaciones con potencias y notaci\'on cient\'ifica,
respetando la jerarqu\'ia de las operaciones. Recuerda que para que tu trabajo
tenga validez deber\'as incluir todas las operaciones y procedimientos que
realices.

\vspace{3mm}
\section{Potencias} % (fold)
\label{sec:potencias}

% section potencias (end)

\begin{equation}
3^{4} + 2^{5} + 4^{3} =
\end{equation}

\vspace{1cm}
\begin{equation}
- (3^{4}) - (2^{5}) - (4^{3}) =
\end{equation}

\vspace{1cm}
\begin{equation}
\left( \frac{2}{5}\right)^{2}  + \frac{1^{10}}{2^{3}}=
\end{equation}

\vspace{1cm}
\begin{equation}
(10-3)^{2} - 6 =
\end{equation}

\vspace{1cm}
\begin{equation}
4.2^{2} \times 0.1^{3}
\end{equation}


\newpage
\section{Notaci\'on cient\'ifica} % (fold)
 \label{sec:notaci'on_cient'ifica}
 
 % section notaci'on_cient'ifica (end) 

Convierte de notaci\'on desarrolla a cient\'ifica y viceversa.

\begin{equation}
7583 = 
\end{equation}

\vspace{1cm}
\begin{equation}
1.78 \times 10^{5}=
\end{equation}

\vspace{1cm}
\begin{equation}
0.000001575=
\end{equation}

\vspace{1cm}
\begin{equation}
4.8764 \times 10^{-7} =
\end{equation}

\vspace{2cm}
Resuelve la siguiente operaci\'on. Represnta el resultado final utilizando
notaci\'on cient\'ifica.

\begin{equation}
4.5 \times 10 ^{5} - 1.2 \times 10^{-3} =
\end{equation}


\end{document}










