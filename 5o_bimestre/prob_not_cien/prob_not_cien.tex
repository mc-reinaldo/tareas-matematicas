\newcommand{\seccion}{SECUNDARIA INCORPORADA A LA SEG }
\newcommand{\descripcion}{\sc Operaciones y problemas con \\notaci\'on cient\'ifica}
\newcommand{\grado}{Primero de Secundaria}
\newcommand{\ciclo}{Ciclo escolar: 2015--2016}
\newcommand{\papel}{letterpaper} %letterpaper, legalpaper ...
\newcommand{\fecha}{3 de junio de 2016}

\author{M. en C. Reinaldo Zapata}

\documentclass[11pt]{article}
\usepackage[\papel]{geometry}

\title{\vspace{-1cm}\flushleft \seccion \\ \descripcion \\  \grado \\ \ciclo}

\newcommand\BackgroundLogo{
\put(160,285){
\parbox[b][\paperheight]{\paperwidth}{%
\vfill
\centering
\includegraphics[width=5cm,height=2.5cm,keepaspectratio]{/Users/reinaldo/Documents/clases/jassa/logo}%
\vfill
}}}

% \hyphenation{con-ti-nua-ci\'on}

\usepackage{enumitem}
\usepackage[T1]{fontenc} %fuentes
\usepackage{lmodern} %fuente mejorada
\usepackage[spanish]{babel}
\decimalpoint
\usepackage{fullpage}
\usepackage{graphicx}
\usepackage{eso-pic}
\usepackage{multirow}
\usepackage{subfigure}
\usepackage{tikz}
\usepackage{hyperref} 
\usepackage{color}
\usepackage{multicol}
\usepackage{tikz}
\usetikzlibrary{shapes.geometric}



\usepackage[leqno,fleqn]{amsmath}
\makeatletter
  \def\tagform@#1{\maketag@@@{#1\@@italiccorr}}
\makeatother
\renewcommand{\theequation}{\fbox{\textbf{\arabic{equation}}}}


\begin{document}
\AddToShipoutPicture*{\BackgroundLogo}
\ClearShipoutPicture
\date{\fecha}
\maketitle

\vspace{-3mm}
% \begin{minipage}[t]{0.8\linewidth}
Nombre del alumno:\,\line(1,0){244}\,.\hspace*{.2cm} \hfill Aciertos:\,\line(1,0){35}\,. \\
\indent Primero de secundaria, grupo:\,\line(1,0){35}\,. No. de lista:\,\line(1,0){35}\,. \hfill 10 \quad \ 
% \end{minipage}
% \begin{minipage}{0.8\linewidth}
% \end{minipage}

\vspace{3mm}
Recuerda que para que tu trabajo tenga validez deber\'as incluir todas las
operaciones y procedimientos que realices.

\section*{Operaciones} % (fold)
\label{sec:operaciones}

Resuelve las siguientes operaciones usnado notaci\'on cient\'ifica. Escribe el
resultado final utilizando s\'olo una posici\'on en los enteros.
\begin{equation}
1.54 \times 10^{-3} + 1.5 \times 10^{-4} + 0.6  \times 10^{-5} = 
\end{equation} 

\vspace{2.5cm}
\begin{equation}
109.5 \times 10^{5} - 1.8 \times 10^{6} - 13.84 \times 10^{3} =
\end{equation}

\vspace{2.5cm}
\begin{equation}
1.5 \times 10^{5} - 67.3 \times 10^{6} + 567  \times 10^{4} = 
\end{equation}

\newpage
% \begin{equation}
% (7.85 \times 10^{15})(2.9 \times 10^{-3}) =
% \end{equation}

\vspace{2.5cm}
\begin{equation}
234 \times 10^{-6} \div 1.4 \times 10^{8} =
\end{equation}

\vspace{2.5cm}
\begin{equation}
(1.3 \times 10^{4})^{3} = \hfill \textbf{(2 aciertos)}º
\end{equation}
% section operaciones (end)

\vspace{2.5cm}
\section*{Problemas} % (fold)
\label{sec:problemas}
Resuelve los siguientes problemas usando notaci\'on cient\'ifica. Escribe el
resultado final utilizando s\'olo una posici\'on en los enteros.

\vspace{3mm}
La f\'ormula para calcular el volumen de una esfera es $v_{e} = \frac{4}{3} \pi
r^{3}$. Utilizando esta f\'ormula calcula el volumen de un \'atomo de
hidr\'ogeno si se tiene que su radio es aproximadamente $5.3 \times
10^{-11}$\,m. Utiliza $\pi = 3.14$.  \hfill \textbf{(2 aciertos)}

\vspace{4cm}
Un granjero tiene un terreno rectangular con medidas $b=11.5 \times 10^{4}$\,m y
$h=9.3 \times 10^{4}$\,m. Calcula el per\'imetro y el \'area de dicho terreno. 
\hfill \textbf{(2 aciertos)} 

% section problemas (end)
\end{document}










